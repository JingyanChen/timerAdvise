%% Generated by Sphinx.
\def\sphinxdocclass{report}
\documentclass[letterpaper,10pt,english]{sphinxmanual}
\ifdefined\pdfpxdimen
   \let\sphinxpxdimen\pdfpxdimen\else\newdimen\sphinxpxdimen
\fi \sphinxpxdimen=.75bp\relax
%% turn off hyperref patch of \index as sphinx.xdy xindy module takes care of
%% suitable \hyperpage mark-up, working around hyperref-xindy incompatibility
\PassOptionsToPackage{hyperindex=false}{hyperref}

\PassOptionsToPackage{warn}{textcomp}

\catcode`^^^^00a0\active\protected\def^^^^00a0{\leavevmode\nobreak\ }
\usepackage{cmap}
\usepackage{xeCJK}
\usepackage{amsmath,amssymb,amstext}
\usepackage{babel}



\setmainfont{FreeSerif}[
  Extension      = .otf,
  UprightFont    = *,
  ItalicFont     = *Italic,
  BoldFont       = *Bold,
  BoldItalicFont = *BoldItalic
]
\setsansfont{FreeSans}[
  Extension      = .otf,
  UprightFont    = *,
  ItalicFont     = *Oblique,
  BoldFont       = *Bold,
  BoldItalicFont = *BoldOblique,
]
\setmonofont{FreeMono}[
  Extension      = .otf,
  UprightFont    = *,
  ItalicFont     = *Oblique,
  BoldFont       = *Bold,
  BoldItalicFont = *BoldOblique,
]


\usepackage[Sonny]{fncychap}
\ChNameVar{\Large\normalfont\sffamily}
\ChTitleVar{\Large\normalfont\sffamily}
\usepackage{sphinx}

\fvset{fontsize=\small}
\usepackage{geometry}


% Include hyperref last.
\usepackage{hyperref}
% Fix anchor placement for figures with captions.
\usepackage{hypcap}% it must be loaded after hyperref.
% Set up styles of URL: it should be placed after hyperref.
\urlstyle{same}
\addto\captionsenglish{\renewcommand{\contentsname}{正文:}}

\usepackage{sphinxmessages}
\setcounter{tocdepth}{1}



\title{timerAdvise}
\date{2020 年 05 月 06 日}
\release{0.1}
\author{JingyanChen}
\newcommand{\sphinxlogo}{\vbox{}}
\renewcommand{\releasename}{发布}
\makeindex
\begin{document}

\ifdefined\shorthandoff
  \ifnum\catcode`\=\string=\active\shorthandoff{=}\fi
  \ifnum\catcode`\"=\active\shorthandoff{"}\fi
\fi

\pagestyle{empty}
\sphinxmaketitle
\pagestyle{plain}
\sphinxtableofcontents
\pagestyle{normal}
\phantomsection\label{\detokenize{index::doc}}



\chapter{STM32的定时器设计介绍}
\label{\detokenize{STM32_u5b9a_u65f6_u5668_u8bbe_u8ba1_u4ecb_u7ecd/STM32_u7684_u5b9a_u65f6_u5668_u8bbe_u8ba1_u4ecb_u7ecd:stm32}}\label{\detokenize{STM32_u5b9a_u65f6_u5668_u8bbe_u8ba1_u4ecb_u7ecd/STM32_u7684_u5b9a_u65f6_u5668_u8bbe_u8ba1_u4ecb_u7ecd::doc}}
以 \sphinxhref{https://www.st.com/resource/en/reference\_manual/cd00171190-stm32f101xx-stm32f102xx-stm32f103xx-stm32f105xx-and-stm32f107xx-advanced-armbased-32bit-mcus-stmicroelectronics.pdf}{STM32F103RC} 为例,ST定义了三种定时器类型。
\begin{itemize}
\item {} 
Basic timers(基础定时器)

\item {} 
General\sphinxhyphen{}purpose timers(通用目的定时器)

\item {} 
Advanced\sphinxhyphen{}control timers(增强型定时器)

\end{itemize}


\section{Basic timers(基础定时器)}
\label{\detokenize{STM32_u5b9a_u65f6_u5668_u8bbe_u8ba1_u4ecb_u7ecd/STM32_u7684_u5b9a_u65f6_u5668_u8bbe_u8ba1_u4ecb_u7ecd:basic-timers}}

\subsection{1. Feature}
\label{\detokenize{STM32_u5b9a_u65f6_u5668_u8bbe_u8ba1_u4ecb_u7ecd/STM32_u7684_u5b9a_u65f6_u5668_u8bbe_u8ba1_u4ecb_u7ecd:feature}}\begin{itemize}
\item {} 
16\sphinxhyphen{}bit auto\sphinxhyphen{}reload upcounter

\item {} 
16\sphinxhyphen{}bit 分频器

\item {} 
与DAC共享部分电路以驱动DAC

\item {} 
计数超时会产生DMA Request/Interrupt

\end{itemize}


\subsection{2. Block Diagram}
\label{\detokenize{STM32_u5b9a_u65f6_u5668_u8bbe_u8ba1_u4ecb_u7ecd/STM32_u7684_u5b9a_u65f6_u5668_u8bbe_u8ba1_u4ecb_u7ecd:block-diagram}}
\sphinxincludegraphics{{BasicTimer}.PNG}

基础定时器提供了基本的功能,外部时钟在分频之后进入到基础定时器中,当基础定时器的CNT计数达到Auto\sphinxhyphen{}reload Register数值的时候
产生中断/DMA请求。基础定时器仅支持上升计数模式,当达到ARR的时候自动清零。


\subsection{3. Register Design}
\label{\detokenize{STM32_u5b9a_u65f6_u5668_u8bbe_u8ba1_u4ecb_u7ecd/STM32_u7684_u5b9a_u65f6_u5668_u8bbe_u8ba1_u4ecb_u7ecd:register-design}}\begin{itemize}
\item {} \begin{description}
\item[{TIMx\_CR1}] \leavevmode\begin{itemize}
\item {} 
ARPE: 选择当溢出事件产生是否重新装载ARR的值

\item {} 
OPM:  是否开启one\sphinxhyphen{}pulse模式,决定溢出事件产生后是否停止计数

\item {} 
URS:  选择触发interrupt/DMA request的事件,是仅overflow/underflow事件还是包含其他

\item {} 
UDIS: interrupt/DMA request事件的使能位

\item {} 
CEN:  计数使能

\end{itemize}

\end{description}

\item {} \begin{description}
\item[{TIMx\_CR2}] \leavevmode\begin{itemize}
\item {} 
MMS:  当基础定时器作为主定时器为从模块提供时钟的时候,选择时钟事件,是Reset事件或者是CNT\_EN事件或者是分频器update事件

\end{itemize}

\end{description}

\item {} \begin{description}
\item[{TIMx\_DIER}] \leavevmode\begin{itemize}
\item {} 
UDE/UIE: interrupt以及DMA的使能

\end{itemize}

\end{description}

\item {} \begin{description}
\item[{TIMx\_SR}] \leavevmode\begin{itemize}
\item {} 
UIF: 中断标志位

\end{itemize}

\end{description}

\item {} \begin{description}
\item[{TIMx\_EGR}] \leavevmode\begin{itemize}
\item {} 
UG:清理CNT的数据,重新计数

\end{itemize}

\end{description}

\item {} \begin{description}
\item[{TIMx\_CNT}] \leavevmode\begin{itemize}
\item {} 
CNT:计数值

\end{itemize}

\end{description}

\item {} \begin{description}
\item[{TIMx\_PSC}] \leavevmode\begin{itemize}
\item {} 
PSC{[}15:0{]}: 分频值

\end{itemize}

\end{description}

\item {} \begin{description}
\item[{TIMx\_ARR}] \leavevmode\begin{itemize}
\item {} 
ARR: 预加载值

\end{itemize}

\end{description}

\end{itemize}


\subsection{4. Application}
\label{\detokenize{STM32_u5b9a_u65f6_u5668_u8bbe_u8ba1_u4ecb_u7ecd/STM32_u7684_u5b9a_u65f6_u5668_u8bbe_u8ba1_u4ecb_u7ecd:application}}\begin{itemize}
\item {} 
提供固定事件的中断信号

\item {} 
为DAC/ADC等其他模块提供工作时钟

\end{itemize}


\subsection{5. 优势与我们可借鉴的方面}
\label{\detokenize{STM32_u5b9a_u65f6_u5668_u8bbe_u8ba1_u4ecb_u7ecd/STM32_u7684_u5b9a_u65f6_u5668_u8bbe_u8ba1_u4ecb_u7ecd:id1}}\begin{itemize}
\item {} 
此模块比较简单,仅支持上升计数模式,仅有一个比较值,产生固定时间的计数信号,可以把这种简单的基础计数器模块当作一个其他设备的时钟分频器。

\item {} 
我们可以借鉴的点是,是否可以设计这种基础计数器为其他的外设提供更加灵活的时钟?

\item {} 
之前ADC的模块存在一个问题,ADC的输入时钟在BL602设计里面只支持1 4 8 16 256 等分频选项,导致了如果输入的时钟不是由专门的PLL时钟电路产生的2.048M时钟,那么mic应用会拿不到标准的8K采样率,因为ADC会做一个64,128,或256次平均,相当于分频,使得最终的ADC时钟取不到整数,如果TIMER模块支持Basic Timer 并且这个Basic Timer可以成为ADC的输入时钟,那么就可以解决这个问题。

\item {} 
如果Basic Timer可以为DAC提供时钟,那么DAC就可以灵活的解调各种采样率的音频。

\end{itemize}


\section{General\sphinxhyphen{}purpose timers(通用目的定时器)}
\label{\detokenize{STM32_u5b9a_u65f6_u5668_u8bbe_u8ba1_u4ecb_u7ecd/STM32_u7684_u5b9a_u65f6_u5668_u8bbe_u8ba1_u4ecb_u7ecd:general-purpose-timers}}

\subsection{1. Feature}
\label{\detokenize{STM32_u5b9a_u65f6_u5668_u8bbe_u8ba1_u4ecb_u7ecd/STM32_u7684_u5b9a_u65f6_u5668_u8bbe_u8ba1_u4ecb_u7ecd:id2}}

\subsection{2. Block Diagram}
\label{\detokenize{STM32_u5b9a_u65f6_u5668_u8bbe_u8ba1_u4ecb_u7ecd/STM32_u7684_u5b9a_u65f6_u5668_u8bbe_u8ba1_u4ecb_u7ecd:id3}}

\subsection{3. Application}
\label{\detokenize{STM32_u5b9a_u65f6_u5668_u8bbe_u8ba1_u4ecb_u7ecd/STM32_u7684_u5b9a_u65f6_u5668_u8bbe_u8ba1_u4ecb_u7ecd:id4}}

\subsection{4. Register Design}
\label{\detokenize{STM32_u5b9a_u65f6_u5668_u8bbe_u8ba1_u4ecb_u7ecd/STM32_u7684_u5b9a_u65f6_u5668_u8bbe_u8ba1_u4ecb_u7ecd:id5}}

\section{Advanced\sphinxhyphen{}control timers(增强型定时器)}
\label{\detokenize{STM32_u5b9a_u65f6_u5668_u8bbe_u8ba1_u4ecb_u7ecd/STM32_u7684_u5b9a_u65f6_u5668_u8bbe_u8ba1_u4ecb_u7ecd:advanced-control-timers}}

\subsection{1. Feature}
\label{\detokenize{STM32_u5b9a_u65f6_u5668_u8bbe_u8ba1_u4ecb_u7ecd/STM32_u7684_u5b9a_u65f6_u5668_u8bbe_u8ba1_u4ecb_u7ecd:id6}}

\subsection{2. Block Diagram}
\label{\detokenize{STM32_u5b9a_u65f6_u5668_u8bbe_u8ba1_u4ecb_u7ecd/STM32_u7684_u5b9a_u65f6_u5668_u8bbe_u8ba1_u4ecb_u7ecd:id7}}

\subsection{3. Application}
\label{\detokenize{STM32_u5b9a_u65f6_u5668_u8bbe_u8ba1_u4ecb_u7ecd/STM32_u7684_u5b9a_u65f6_u5668_u8bbe_u8ba1_u4ecb_u7ecd:id8}}

\subsection{4. Register Design}
\label{\detokenize{STM32_u5b9a_u65f6_u5668_u8bbe_u8ba1_u4ecb_u7ecd/STM32_u7684_u5b9a_u65f6_u5668_u8bbe_u8ba1_u4ecb_u7ecd:id9}}


\renewcommand{\indexname}{索引}
\printindex
\end{document}